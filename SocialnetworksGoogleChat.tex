\documentclass[hyperref={pdfpagelabels=false}]{beamer}
%\documentclass{beamer}
%\documentclass[draft]{beamer}

\usepackage[german]{babel}
\usepackage[utf8]{inputenc}
\usepackage{url}
\usepackage{endnotes}
\usepackage{calc}

\usetheme{Antibes}
\usecolortheme{dolphin}
\usefonttheme{default}
\setbeamertemplate{sidebar left}{\vfill\hspace{0.1cm}\pgfimage[height=.5cm]{bilder/CC-BY-NC-SA}}
\setbeamertemplate{footline}[frame number]
\setbeamercovered{transparent}

% Fixing Warnings
\RequirePackage{fix-cm}
\RequirePackage{pgfcore}
\usepackage{pgfpages} 
\let\Tiny=\tiny
\usepackage{lmodern}

\newcommand{\email}[2]{#1@#2}
%\def\supertiny{ \font\supertinyfont = cmr10 at 1pt \relax \supertinyfont} 
\renewcommand{\enotesize}{\tiny}

\title{Facebook, Google, Twitter \& Co.}
\subtitle{Freud und Leid mit den Gefahren im sozialen Netz}
\author{%
		Stefan Schlott\inst{1}\and%
		Andreas Herz\inst{2}%
}
\institute{%
	\inst{1}\email{stefan.schlott}{ulm.ccc.de}\and%
	\inst{2}\email{andi}{geekosphere.org}%
}
\date{03.05.2010}
\logo{\vspace{5.5cm} \pgfimage[height=2cm]{bilder/Chaosknoten}}


\begin{document}

\begin{frame}[plain]
\titlepage
\end{frame}

\begin{frame}{Über uns...}
	\begin{columns}[t]
		\column{.53\textwidth}
			Stefan
			\begin{itemize}
				\item Informatikstudium an der Uni Ulm
				\item Promoviert im Themenbereich Privacy
				\item Mitglied beim CCC-Erfa-Kreis Ulm (\url{http://ulm.ccc.de/})
				\item Bloggt unter \\ \url{http://stefan.ploing.de/}
			\end{itemize}
		\column{.47\textwidth}
			Andreas
			\begin{itemize}
				\item Student an der Uni Augsburg
				\item 8. Semester Diplom Informatik
				\item CCC-Mitglied vom Chaostreff Augsburg (\url{http://www.c3a.de/})
			\end{itemize}
	\end{columns}
	\vfill
\end{frame}

\begin{frame}{Der CCC - Chaos Computer Club}
	\begin{quotation}\small
		Der Chaos Computer Club ist eine galaktische Gemeinschaft von Lebewesen, unabhängig von Alter, Geschlecht und Abstammung sowie gesellschaftlicher Stellung, die sich grenzüberschreitend für Informationsfreiheit einsetzt und mit den Auswirkungen von Technologien auf die Gesellschaft sowie das einzelne Lebewesen beschäftigt und das Wissen um diese Entwicklung fördert.
	\end{quotation}

	\begin{itemize}
		\item Größte europäische Hackervereinigung
		\item Seit über 25 Jahren Vermittler im Spannungsfeld technischer und sozialer Entwicklungen
		\item Mischung aus Hackern und Computerspezialisten -- aber auch Datenschützern, Gesellschaftskritiker und Künstler
	\end{itemize}
\end{frame}

\begin{frame}{Hacker und Hackerethik}
	\begin{block}{Hacker: Kreativer Ge- oder Mißbrauch von Technik}
		\begin{quotation}
			Jemand, der mit einer Kaffeemaschine Toast zubereiten kann.
		\end{quotation}
	\end{block}

	\begin{block}{Hackerethik:}
		\begin{quotation}
			Öffentliche Daten nützen, private Daten schützen.
		\end{quotation}
	\end{block}
\end{frame}

\begin{frame}{Marc L. - Ein Google-Portrait\footnote{http://stefan.ploing.de/2009-05-13-marc-l-google-portrait}}
	\begin{itemize}
		\item Ein Experiment des frz. Magazins "`Le Tigre"'
		\item<2-> Reportage über eine zufällig gewählte Person
		\item<2-> Infos aus flickr, Facebook, Stayfriends
		\item<3-> Hobbies, Urlaube, Arbeitsplätze
		\item<3-> Wohnort, Telefon, Handy
		\item<3-> Beziehungen (incl. "`Partnerwechsel"')
		\item<4-> Veröffentlichung im Print-Magazin mit vollen Namen
		\item<4-> "`Belegexemplar"' direkt an Marc L. geschickt
	\end{itemize}
\end{frame}

\begin{frame}{Wenn die Welt mitliest}
	\begin{columns}
		\column{.5\textwidth}
		\pgfimage[height=2cm]{bilder/facebook-fail-texting}

		Ups. Facebook im Unterricht, Lehrer liest mit.
		\column{.5\textwidth}
		\pgfimage[height=3.4cm]{bilder/facebook-fail-fired}

		Ups! Auskotzen über den Chef, Chef liest mit. Kündigung.
	\end{columns}
\end{frame}

\begin{frame}{Risiken und Nebenwirkungen...}
	\begin{itemize}
		\item Voreingenommene Gesprächspartner (Saufbilder und Bewerbungen, etc.)
		\item Mobbing
		\item Perfekte Quelle für Social Engineering
		\begin{itemize}
			\item<2-> Die "`menschliche Firewall"' hacken
			\item<3-> Opfer werden: Gezielte Infos für Täter, wie er manipulieren kann (Hobbies, Schwächen, mögliche Druckmittel)
			\item<3-> Infoquelle: Opfer im Umfeld, eigenes Profil reichert Wissen des Täters an (Täter kann besser vorgeben, zum Umfeld zu gehören)
		\end{itemize}
	\end{itemize}
\end{frame}

\begin{frame}{"`I need money"' scam}
	\begin{columns}[t]
		\column{.5\textwidth}
		Vom Opfer erlebt:
		\begin{itemize}
			\item Freundin meldet sich via Facebook nach längerer Zeit
			\item Ein wenig Smalltalk
			\item<2-> Hilferuf: Geldbeutel mit Flugticket gestohlen
			\item<2-> Bitte: Geld leihen (per Bankanweisung, z.B. Western Union)
		\end{itemize}
		\column{.5\textwidth}
		Was wirklich passierte:
		\begin{itemize}
			\item<3-> Account von Freundin gehackt (Info von Pinnwand: Urlaub = guter Zeitpunkt)
			\item<3-> ...oder Fake-Account angelegt, "`eingeschlichen"'
			\item<3-> Bankanweisung: Faktisch nicht verfolgbar
		\end{itemize}
	\end{columns}
\end{frame}

\begin{frame}{das sind die sichtbaren Daten... was gibt es noch?}
\end{frame}

\begin{frame}{Was kann man mit solchen Infos machen?}
\end{frame}

\begin{frame}{Daten Löschen?}
	\begin{itemize}
		\item Gespeicherte Kopien bei Betrachtern
		\item Automatisierte Spiegelungen (z.B. archive.org)
		\item Caches (z.B. Google Cache)
		\item Daten gelöscht oder nur unsichtbar?
		\item Stiftung Warentest: Träge Reaktionen auf Löschanfragen
	\end{itemize}
\end{frame}

\begin{frame}{Warum ist ein gewisses Mißtrauen gegenüber den Netzprovidern angebracht?}
\end{frame}

% vim: ai ts=2 sw=2


\begin{frame}
Suchmaschinen
\end{frame}

\begin{frame}{Suchmaschinen}
	\begin{itemize}
		\item Sammeln enorme Mengen an Daten um Suchbegriffe zu optimieren.
		\item Bieten oft weitere Dienste an (z.B. Google Mail).
		\item Datensammlungen auf die jeder Zugriff hat.
	\end{itemize}
\end{frame}

\begin{frame}{Beispiel: AOL-Daten}
	\begin{itemize}
		\item AOL ver\"offentlicht (versehentlich) 2006 Suchanfragen von rund 650.000 Kunden.
		\item Trotz Anonymisierung in Form von einer reinen Nummer konnten Profile erstellt werden.
		\item Mrs. Arnold als Beispiel in der New York Times.%
		\urlref{1}{http://www.nytimes.com/2006/08/09/technology/09aol.html?_r=3&pagewanted=1}
	\end{itemize}
\end{frame}

\begin{frame}{Google Analytics und ivwbox}
	\begin{itemize}
		\item Google Analytics - Google Dienst zur Datensammlung, auf fast allen Webangeboten vertreten
		\item ivwbox.de - Datensammlung der VG Wort
	\end{itemize}
	\begin{block}{Gegenma\ss nahmen}
		NoScript Addon f\"ur den Firefox.
	\end{block}
\end{frame}

\begin{frame}{Datensammler}
	\begin{columns}
		\column{.5\textwidth}
			Wirtschaft:
			\begin{itemize}
				\item Suchmaschinen
				\item SocialNetworks
				\item H\"andler (z.B. Amazon)
				\item Provider
			\end{itemize}
		\column{.5\textwidth}
			Staat:
			\begin{itemize}
				\item Vorratsdatenspeicherung
				\item Elektronische Gesundheitskarte
				\item Reisepass und Personalausweis (RFID)
				\item SWIFT Abkommen
			\end{itemize}
	\end{columns}
\end{frame}
\begin{frame}{Interessen der Sammler}
	\begin{itemize}
		\item Handel mit Daten (Wirtschaft)
		\item \"Uberwachung
		\item Scoring
		\item Datendiebstahl
		\item Profilerstellung
		\item Werbung
		\item Verbreitung von Schadsoftware (Malware)
	\end{itemize}
\end{frame}
\begin{frame}{Auskunftsanspruch}
	\begin{itemize}
		\item Recht auf Auskunft \"uber gespeicherte Daten
		\item Recht auf L\"oschung bestimmter Daten
		\item Widerspruchsrecht
		\item Opt-in und Opt-out (z.B. Meldeamt)
	\end{itemize}
	\begin{block}{Aufruf:}
		Macht von euren Rechten regen Gebrauch!
	\end{block}
\end{frame}


% vim: ai ts=2 sw=2


\begin{frame}{Chats}
\end{frame}

\begin{frame}{AGBs gelesen?}
Comic: "`What do you remember from the terms of service?"' - "`I accept"'

Aprilscherz "`Seelenverkauf"' in den AGBs
\end{frame}
\begin{frame}{ICQ AGB}
	\begin{quotation}
		"You agree that by posting any material or information anywhere on the ICQ Services 
		and Information you surrender your copyright and any other proprietary right in the
		posted material or information. You further agree that ICQ Inc. is entitled to use at 
		its own discretion any of the posted material or information in any manner it deems fit, 
		including, but not limited to, publishing the material or distributing it.
	\end{quotation}

\end{frame}

\begin{frame}{Keine gesicherten Identitäten}
	\begin{itemize}
		\item Pseudonyme und keine echten Namen
		\item Kaum bis keine Kontrolle (Jugendschutz)
		\item Falsche Identit\"aten
	\end{itemize}
\end{frame}

\begin{frame}{Alternativen}
	\begin{itemize}
		\item Jabber/XMPP: Freies Protokoll, dezentraler und viele Clients
		\item Accountname George@orwell.jabber.org
		\item Wird bei Lokalisten.de, web.de, gmx.de, google.de... verwendet
		\item IRC: Internet Relay Chat
	\end{itemize}
\end{frame}

% vim: ai ts=2 sw=2


\begin{frame}{Maßnahmen}
	\begin{itemize}
		\item Welche M\"oglichkeiten hat man als Nutzer?
		\item Rechtliche Aspekte
		\item Soziale Gesichtspunkte
		\item Technische Hinweise
	\end{itemize}
\end{frame}

\begin{frame}{Datenschutz}
	\begin{center}Ist Datenschutz wichtig?
	\end{center}
	\begin{itemize}
		\item Allgemeines Pers\"onlichkeitsrecht (z.B. Art. 2 Abs. I GG)
		\item Grundrecht auf informationelle Selbstbestimmung (BVerfGE 65, 1 –Volksz\"ahlungsurteil)
		\item Datenschutzgesetze (Gesetze von Bund und Land)
	\end{itemize}
	\begin{itemize}
		\item L\"oschen von Daten schwierig.
		\item Verkn\"upfung von Daten problematisch
		\item Unterschiedliche Rechte in den L\"andern
		\item Problematische AGBs
	\end{itemize}
\end{frame}

\begin{frame}{Datensparsamkeit}
	\begin{itemize}
		\item Erst nachdenken, dann Daten preisgeben!
		\item Wem gebe ich meine Daten?
		\item Wie sind meine Daten gesichert?
		\item Welche Einflu\ss m\"oglichkeiten hab ich?
		\item Daten sind meist von allen lesbar, auch Lehrern!
		\item Das Internet vergisst nichts, auch nicht nach 10 oder 20 Jahren.
	\end{itemize}
\end{frame}

\begin{frame}{Weitere grundlegende Vorkehrungen}
	\begin{itemize}
		\item Verwendet Nicknames bzw. Rollenpseudonyme
		\item Vollst\"andigen Namen gek\"urzt
		\item Unterschiedliche Accounts
		\item Nicht die gleichen Passw\"orter verwenden!\urlref{1}{http://stefan.ploing.de/2007-03-15-mittel-gegen-passwort-inflation}
		\item Keine weiteren Infos wie Geburtsjahr in den Nickname
	\end{itemize}
\end{frame}

\begin{frame}{''Freundschaften?!''}
	\begin{itemize}
		\item Nicht jeder ist dein Freund
		\item Freundesketten erlauben umfangreichen Datenzugriff
		\item Mobbing?! (schwer technisch zu l\"osen)
		\item Partyfotos und Pers\"onlichkeitsrecht
		\item Freunde Highscore = uncool!
	\end{itemize}
\end{frame}


\begin{frame}{Cookies begrenzen}
	\begin{itemize}
		\item Browsereinstellungen oder Addons f\"ur Firefox.
		\item Cookies nur gezielt zulassen oder Session-Cookies verwenden bzw. Cookies l\"oschen.
	\end{itemize}
\end{frame}

\begin{frame}{Vertrauenswürdige Infrastruktur}
	\begin{itemize}
		\item Jabber-Server statt ICQ/MSN etc.
		\item OTR/GPG zur Verschl\"usselung
		\item SSL/TLS (https) zur Sicherung der Verbindung und Logindaten
		\item Alternative Suchmaschinen (ixquick, metager)
		\item Anonymisierungsdienste wie TOR und Anon bzw. Trashmail Accounts
	\end{itemize}
\end{frame}

\begin{frame}{Ziel}
	\begin{quotation}
		Bewusster und sorgsamer Umgang mit eigenen Daten und deren Anderer, statt Verboten.
	\end{quotation}
\end{frame}


% vim: ai ts=2 sw=2


\begin{frame}[plain]
	That's it!
	\begin{center}
		\pgfimage[height=6cm]{bilder/chaosknoten-fernsehturm}
	\end{center}
\end{frame}

\begin{frame}[plain]
	\theendnotes
\end{frame}

\end{document}
% vim: ai ts=2 sw=2
