\begin{frame}{Marc L. - Ein Google-Portrait\footnote{http://stefan.ploing.de/2009-05-13-marc-l-google-portrait}}
	\begin{itemize}
		\item Ein Experiment des frz. Magazins "`Le Tigre"'
		\item<2-> Reportage über eine zufällig gewählte Person
		\item<2-> Infos aus flickr, Facebook, Stayfriends
		\item<3-> Hobbies, Urlaube, Arbeitsplätze
		\item<3-> Wohnort, Telefon, Handy
		\item<3-> Beziehungen (incl. "`Partnerwechsel"')
		\item<4-> Veröffentlichung im Print-Magazin mit vollen Namen
		\item<4-> "`Belegexemplar"' direkt an Marc L. geschickt
	\end{itemize}
\end{frame}

\begin{frame}{Risiken und Nebenwirkungen...}
	\begin{itemize}
		\item Voreingenommene Gesprächspartner (Saufbilder und Bewerbungen, etc.)
		\item Mobbing
		\item Perfekte Quelle für Social Engineering
	\end{itemize}
\end{frame}

\begin{frame}{das sind die sichtbaren Daten... was gibt es noch?}
\end{frame}

\begin{frame}{Was kann man mit solchen Infos machen?}
\end{frame}

\begin{frame}{Daten Löschen?}
	\begin{itemize}
		\item Gespeicherte Kopien bei Betrachtern
		\item Automatisierte Spiegelungen (z.B. archive.org)
		\item Caches (z.B. Google Cache)
		\item Daten gelöscht oder nur unsichtbar?
		\item Stiftung Warentest: Träge Reaktionen auf Löschanfragen
	\end{itemize}
\end{frame}

\begin{frame}{Warum ist ein gewisses Mißtrauen gegenüber den Netzprovidern angebracht?}
\end{frame}

% vim: ai ts=2 sw=2
