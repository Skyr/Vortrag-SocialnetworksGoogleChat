\begin{frame}{Marc L. - Ein Google-Portrait\footnote{http://stefan.ploing.de/2009-05-13-marc-l-google-portrait}}
	\begin{itemize}
		\item Ein Experiment des frz. Magazins "`Le Tigre"'
		\item<2-> Reportage über eine zufällig gewählte Person
		\item<2-> Infos aus flickr, Facebook, Stayfriends
		\item<3-> Hobbies, Urlaube, Arbeitsplätze
		\item<3-> Wohnort, Telefon, Handy
		\item<3-> Beziehungen (incl. "`Partnerwechsel"')
		\item<4-> Veröffentlichung im Print-Magazin mit vollen Namen
		\item<4-> "`Belegexemplar"' direkt an Marc L. geschickt
	\end{itemize}
\end{frame}

\begin{frame}{Wenn die Welt mitliest}
	\begin{columns}
		\column{.5\textwidth}
		\pgfimage[height=2cm]{bilder/facebook-fail-texting}

		Ups. Facebook im Unterricht, Lehrer liest mit.
		\column{.5\textwidth}
		\pgfimage[height=3.4cm]{bilder/facebook-fail-fired}

		Ups! Auskotzen über den Chef, Chef liest mit. Kündigung.
	\end{columns}
\end{frame}

\begin{frame}{Risiken und Nebenwirkungen...}
	\begin{itemize}
		\item Voreingenommene Gesprächspartner (Saufbilder und Bewerbungen, etc.)
		\item Mobbing
		\item Perfekte Quelle für Social Engineering
		\begin{itemize}
			\item<2-> Die "`menschliche Firewall"' hacken
			\item<3-> Opfer werden: Gezielte Infos für Täter, wie er manipulieren kann (Hobbies, Schwächen, mögliche Druckmittel)
			\item<3-> Infoquelle: Opfer im Umfeld, eigenes Profil reichert Wissen des Täters an (Täter kann besser vorgeben, zum Umfeld zu gehören)
		\end{itemize}
	\end{itemize}
\end{frame}

\begin{frame}{"`I need money"' scam}
	\begin{columns}[t]
		\column{.5\textwidth}
		Vom Opfer erlebt:
		\begin{itemize}
			\item Freundin meldet sich via Facebook nach längerer Zeit
			\item Ein wenig Smalltalk
			\item<2-> Hilferuf: Geldbeutel mit Flugticket gestohlen
			\item<2-> Bitte: Geld leihen (per Bankanweisung, z.B. Western Union)
		\end{itemize}
		\column{.5\textwidth}
		Was wirklich passierte:
		\begin{itemize}
			\item<3-> Account von Freundin gehackt (Info von Pinnwand: Urlaub = guter Zeitpunkt)
			\item<3-> ...oder Fake-Account angelegt, "`eingeschlichen"'
			\item<3-> Bankanweisung: Faktisch nicht verfolgbar
		\end{itemize}
	\end{columns}
\end{frame}

\begin{frame}{das sind die sichtbaren Daten... was gibt es noch?}
\end{frame}

\begin{frame}{Was kann man mit solchen Infos machen?}
\end{frame}

\begin{frame}{Daten Löschen?}
	\begin{itemize}
		\item Gespeicherte Kopien bei Betrachtern
		\item Automatisierte Spiegelungen (z.B. archive.org)
		\item Caches (z.B. Google Cache)
		\item Daten gelöscht oder nur unsichtbar?
		\item Stiftung Warentest: Träge Reaktionen auf Löschanfragen
	\end{itemize}
\end{frame}

\begin{frame}{Warum ist ein gewisses Mißtrauen gegenüber den Netzprovidern angebracht?}
\end{frame}

% vim: ai ts=2 sw=2
