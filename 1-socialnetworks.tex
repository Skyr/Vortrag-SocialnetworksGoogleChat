\begin{frame}{Soziale Netze}
	\begin{columns}
		\column{.5\textwidth}
		\begin{center}\pgfimage[height=8mm]{bilder/logo-facebook}\end{center}

		\begin{center}\pgfimage[height=8mm]{bilder/logo-kwick}\end{center}

		\begin{center}\pgfimage[height=8mm]{bilder/logo-lokalisten}\end{center}

		\begin{center}\pgfimage[height=8mm]{bilder/logo-myspace}\end{center}
		\column{.5\textwidth}
		\begin{center}\pgfimage[height=8mm]{bilder/logo-pausenhof}\end{center}

		\begin{center}\pgfimage[height=8mm]{bilder/logo-schuelervz}\hspace{2mm}
		\pgfimage[height=8mm]{bilder/logo-studivz}\end{center}

		\begin{center}\pgfimage[height=8mm]{bilder/logo-werkenntwen}\end{center}

		\begin{center}\pgfimage[height=8mm]{bilder/logo-xing}\end{center}
	\end{columns}
\end{frame}

\begin{frame}{Soziale Netze}
	\begin{itemize}
		\item Wer von euch ist in einem sozialen Netz (oder mehreren)?
		\item Wie seid ihr dazu gekommen?
		\item Was findet ihr daran cool?
		\item Wofür nutzt ihr es?
	\end{itemize}
\end{frame}

\begin{frame}{Marc L. - Ein Google-Portrait\urlref{1}{http://stefan.ploing.de/2009-05-13-marc-l-google-portrait}}
	\begin{itemize}
		\item Ein Experiment des frz. Magazins "`Le Tigre"'
		\item<2-> Reportage über eine zufällig gewählte Person
		\item<2-> Infos aus flickr, Facebook, Stayfriends
		\item<3-> Hobbies, Urlaube, Arbeitsplätze
		\item<3-> Wohnort, Telefon, Handy
		\item<3-> Beziehungen (incl. "`Partnerwechsel"')
		\item<4-> Veröffentlichung im Print-Magazin mit vollen Namen
		\item<4-> "`Belegexemplar"' direkt an Marc L. geschickt
	\end{itemize}
\end{frame}

\begin{frame}{Wenn die Welt mitliest}
	\begin{columns}
		\column{.5\textwidth}
		\pgfimage[height=2cm]{bilder/facebook-fail-texting}

		Ups. Facebook im Unterricht, Lehrer liest mit.
		\column{.5\textwidth}
		\pgfimage[height=3.4cm]{bilder/facebook-fail-fired}

		Ups! Auskotzen über den Chef, Chef liest mit. Kündigung.
	\end{columns}
\end{frame}

\begin{frame}{Risiken und Nebenwirkungen...}
	\begin{itemize}
		\item Voreingenommene Gesprächspartner (Saufbilder und Bewerbungen, etc.)
		\item Mobbing
		\item Perfekte Quelle für Social Engineering
		\begin{itemize}
			\item<2-> Die "`menschliche Firewall"' hacken
			\item<3-> Opfer werden: Gezielte Infos für Täter, wie er manipulieren kann (Hobbies, Schwächen, mögliche Druckmittel)
			\item<3-> Infoquelle: Opfer im Umfeld, eigenes Profil reichert Wissen des Täters an (Täter kann besser vorgeben, zum Umfeld zu gehören)
		\end{itemize}
	\end{itemize}
\end{frame}

\begin{frame}{"`I need money"' scam}
	\begin{columns}[t]
		\column{.5\textwidth}
		Vom Opfer erlebt\urlref{1}{http://personalweb.about.com/od/makefriendsonfacebook/qt/facebkscammoney.htm}:
		\begin{itemize}
			\item Freundin meldet sich via Facebook nach längerer Zeit
			\item Ein wenig Smalltalk
			\item<2-> Hilferuf: Geldbeutel mit Flugticket gestohlen
			\item<2-> Bitte: Geld leihen (per Postanweisung, z.B. Western Union)
		\end{itemize}
		\column{.5\textwidth}
		Was wirklich passierte:
		\begin{itemize}
			\item<3-> Account von Freundin gehackt (Info von Pinnwand: Urlaub = guter Zeitpunkt)
			\item<3-> ...oder Fake-Account angelegt, "`eingeschlichen"'
			\item<3-> Postanweisung: Faktisch nicht verfolgbar
		\end{itemize}
	\end{columns}
\end{frame}

\begin{frame}{Das sind die sichtbaren Daten... was gibt es noch?}
	Daten aus "`Datenlecks"': Einige wenige Beispiele...
	\begin{itemize}
		\item VZ-Profilseiten automatisiert abfragbar (Crawling, offene Schnittstellen%
		\urlref{1}{http://www.netzpolitik.org/2010/neues-datenleck-bei-schuelervz/}
		\urlref{1}{http://www.coresec.de/index.php/2010/05/04/schuelervz-crawler-lenaml/}
		\urlref{1}{http://www.netzpolitik.org/2009/netzpolitik-interview-sicherheitsluecken-bei-der-vz-gruppe/}
		\urlref{1}{http://blogbar.de/archiv/2007/02/27/oooops-bei-studivz-muss-was-absolut-grobes-passiert-sein/})
		\item Mailadressen aller Facebook-Profile kurzzeitig sichtbar%
		\urlref{1}{http://www.heise.de/newsticker/meldung/Datenpanne-bei-Facebook-968117.html}
		\item Profile und Accounts bei haefft.de zugreifbar%
		\urlref{1}{http://www.ccc.de/de/updates/2009/haefft-datenloch}
	\end{itemize}

	\uncover<2>{
		Daten des Netzwerk-Betreibers
		\begin{itemize}
			\item ...speichert viel mehr, als das, was man sieht!
			\item Facebook: Jeder Klick, jede Seite, jede Veränderung%
			\urlref{2}{http://therumpus.net/2010/01/conversations-about-the-internet-5-anonymous-facebook-employee/?full=yes}
			\item Facebook: Löscht nie, macht nur unsichtbar%
			\urlref{2}{http://www.spiegel.de/netzwelt/web/0,1518,608116,00.html}
			\item Facebook: Verfolgt (selbst nicht angemeldete) Nutzer auch über Website hinaus (mit Facebook-Buttons)%
			\urlref{2}{http://www.golem.de/0912/71670.html}
		\end{itemize}
	}
\end{frame}

\begin{frame}{Was kann man mit solchen Infos machen?}
	\begin{itemize}
		\item Persönlichkeits- und Kommunikationsprofile erstellen
		\item Personalisierte Werbung (Werbebanner, Mail, Post)
		\item Auftragsarbeiten: Statistiken erstellen
		\item Adresshandel
	\end{itemize}
\end{frame}

\begin{frame}{Beispiel: Gecrawlte StudiVZ-Daten 2006}
	\begin{columns}
		\column{.5\textwidth}
		\pgfimage[width=4.7cm]{bilder/studivz-groups-uebersicht}
		\column{.5\textwidth}
		\pgfimage[width=4.7cm]{bilder/studivz-groups-ausschnitt}
	\end{columns}
	\vfill
	{\tiny Quelle: \url{http://studivz.irgendwo.org/}}
\end{frame}

\begin{frame}{Wieso sollte mein Netz so etwas tun?}
	\begin{columns}
		\column{.5\textwidth}
		Ausgaben:
		\begin{itemize}
			\item VZnet: Laut eigener Webseite über 300 Mitarbeiter\urlref{1}{http://www.studivz.net/l/about_us/1/}
			\item Facebook: Rund 850 Mitarbeiter\urlref{1}{http://www.heise.de/newsticker/meldung/Facebook-200-Millionen-Nutzer-und-kein-Geld-212506.html}
			\item<2-> Ausgaben für Personal, Server, Netzanbindung, Werbung, ...
		\end{itemize}
		\column{.5\textwidth}
		Einnahmen:
		\begin{itemize}
			\item<3-> Werbung
			\item<4-> ...aber Werbemarkt in den letzten zwei Jahren eingebrochen (einfache Banner: 75\%!)
			\item<5-> Mitgliedschaft: Kostenlos!
		\end{itemize}
	\end{columns}
\end{frame}

\begin{frame}{Das Geschäft mit den Daten}
	\begin{itemize}
		\item VZnet verkauft personalisierte Bannerwerbung (wird auf ihren Webseiten anschaulich erklärt\urlref{1}{http://www.studivz.net/l/wozu_das_ganze})
		\item<2-> MySpace verkauft "`anonymisierte"', vorausgewertete Daten\urlref{2}{http://winfuture.de/news,54207.html}
		\item<3-> ...und Facebook möchte mit seinen Daten sowieso tun und lassen, was es will:
	\end{itemize}
	\uncover<4->{
	\begin{quotation}\footnotesize
		"`ein unwiderrufliches, fortwährendes, nicht-exklusives, übertragbares, voll bezahltes, weltweites Recht"', alle Benutzerinhalte Inhalte zu "`nutzen, kopieren, veröffentlichen, streamen, speichern, öffentlich aufführen oder zeigen, übertragen, scannen, neu formatieren, modifizieren, bearbeiten, gestalten, übersetzen, zitieren, adaptieren, neue Inhalte daraus ableiten und verbreiten"' zu dürfen. Dieses Recht gilt auch für eine "`kommerzielle Nutzung und Werbung"' auf der Plattform selbst, aber auch außerhalb.
		\urlref{4}{http://www.golem.de/0902/65345.html}
	\end{quotation}
	}
\end{frame}

\begin{frame}{Daten Löschen?}
	\begin{itemize}
		\item Gespeicherte Kopien bei Betrachtern
		\item Automatisierte Spiegelungen (z.B. archive.org)
		\item Caches (z.B. Google Cache)
		\item Daten gelöscht oder nur unsichtbar?
		\item Stiftung Warentest: Träge Reaktionen auf Löschanfragen%
		\endnote{\url{http://www.test.de/themen/computer-telefon/test/Soziale-Netzwerke-Datenschutz-oft-mangelhaft-1854798-1855785/}}
	\end{itemize}
\end{frame}

% vim: ai ts=2 sw=2
