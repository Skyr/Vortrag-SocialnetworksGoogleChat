\begin{frame}
Maßnahmen
\end{frame}

\begin{frame}{Maßnahmen}
	\begin{itemize}
		\item Welche M\"oglichkeiten hat man als Nutzer?
		\item Rechtliche Aspekte
		\item Soziale Gesichtspunkte
		\item Technische Hinweise
	\end{itemize}
\end{frame}

\begin{frame}{Datenschutz}
	\begin{center}Ist Datenschutz wichtig?
	\end{center}
	\begin{itemize}
		\item Allgemeines Pers\"onlichkeitsrecht (z.B. Art. 2 Abs. I GG)
		\item Grundrecht auf informationelle Selbstbestimmung (BVerfGE 65, 1 –Volksz\"ahlungsurteil)
		\item Datenschutzgesetze (Gesetze von Bund und Land)
	\end{itemize}
	\begin{itemize}
		\item L\"oschen von Daten schwierig.
		\item Verkn\"upfung von Daten problematisch
		\item Unterschiedliche Rechte in den L\"andern
		\item Problematische AGBs
	\end{itemize}
\end{frame}

\begin{frame}{Datensparsamkeit}
	\begin{itemize}
		\item Erst nachdenken, dann Daten preisgeben!
		\item Wem gebe ich meine Daten?
		\item Wie sind meine Daten gesichert?
		\item Welche Einflu\ss m\"oglichkeiten hab ich?
		\item Daten sind meist von allen lesbar, auch Lehrern!
		\item Das Internet vergisst nichts, auch nicht nach 10 oder 20 Jahren.
	\end{itemize}
\end{frame}

\begin{frame}{Weitere grundlegende Vorkehrungen}
	\begin{itemize}
		\item Verwendet Nicknames
		\item Keine weiteren Infos wie Alter oder Geburtsjahr in den Nickname
		\item Wenn kein Nickname möglich: Vollständigen Namen gekürzt
		\item Rollenpseudonyme
		\begin{itemize}
			\item Für jede Aktivität (Hobby, etc.): Eigener Nickname
			\item Wenn nötig: Mehrere getrennte Accounts auf selbem System
			\item Fremde können so nur einen Teilaspekt des eigenen Lebens zusammenführen
		\end{itemize}
		\item Nicht die gleichen Passw\"orter verwenden!\urlref{1}{http://stefan.ploing.de/2007-03-15-mittel-gegen-passwort-inflation}
	\end{itemize}
\end{frame}

\begin{frame}{''Freundschaften?!''}
	\begin{itemize}
		\item Nicht jeder ist dein Freund
		\item Freundesketten erlauben umfangreichen Datenzugriff
		\item Mobbing?! (schwer technisch zu l\"osen)
		\item Partyfotos und Pers\"onlichkeitsrecht
		\item Freunde Highscore = uncool!
	\end{itemize}
\end{frame}


\begin{frame}{Cookies begrenzen}
	\begin{itemize}
		\item Browsereinstellungen oder Addons f\"ur Firefox.
		\item Cookies nur gezielt zulassen oder Session-Cookies verwenden bzw. Cookies l\"oschen.
	\end{itemize}
\end{frame}

\begin{frame}{Vertrauenswürdige Infrastruktur}
	\begin{itemize}
		\item Jabber-Server statt ICQ/MSN etc.
		\item OTR/GPG zur Verschl\"usselung
		\item SSL/TLS (https) zur Sicherung der Verbindung und Logindaten
		\item Alternative Suchmaschinen (ixquick, metager)
		\item Anonymisierungsdienste wie TOR und Anon bzw. Trashmail Accounts
	\end{itemize}
\end{frame}

\begin{frame}{Ziel}
	\begin{quotation}
		Bewusster und sorgsamer Umgang mit eigenen Daten und deren Anderer, statt Verboten.
	\end{quotation}
\end{frame}


% vim: ai ts=2 sw=2
